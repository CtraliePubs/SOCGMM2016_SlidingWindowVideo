\documentclass[a4paper,UKenglish]{lipics}
%This is a template for producing LIPIcs articles. 
%See lipics-manual.pdf for further information.
%for A4 paper format use option "a4paper", for US-letter use option "letterpaper"
%for british hyphenation rules use option "UKenglish", for american hyphenation rules use option "USenglish"
% for section-numbered lemmas etc., use "numberwithinsect"
 
\usepackage{microtype}%if unwanted, comment out or use option "draft"

%\graphicspath{{./graphics/}}%helpful if your graphic files are in another directory

\bibliographystyle{plain}% the recommended bibstyle

% Author macros::begin %%%%%%%%%%%%%%%%%%%%%%%%%%%%%%%%%%%%%%%%%%%%%%%%
\title{The Geometry of Sliding Windows of Periodic Videos}

\author[1]{Christopher J. Tralie}
\affil[1]{Department of Electrical and Computer Engineering, Duke University\\
  Durham, NC USA
  \texttt{chris.tralie@gmail.com}}
\authorrunning{C.\,J Tralie} %mandatory. First: Use abbreviated first/middle names. Second (only in severe cases): Use first author plus 'et. al.'

\Copyright{Christopher J. Tralie}%mandatory, please use full first names. LIPIcs license is "CC-BY";  http://creativecommons.org/licenses/by/3.0/

\subjclass{I.2.10 Video Analysis, I.5.4 Pattern Recognition: Waveform Analysis, I.4.10 Image Representation: Multidimensional}% mandatory: Please choose ACM 1998 classifications from http://www.acm.org/about/class/ccs98-html . E.g., cite as "F.1.1 Models of Computation". 
\keywords{Video Processing, High Dimensional Geometry, Circular Coordinates, Nonlinear Time Series}% mandatory: Please provide 1-5 keywords
% Author macros::end %%%%%%%%%%%%%%%%%%%%%%%%%%%%%%%%%%%%%%%%%%%%%%%%%

%Editor-only macros:: begin (do not touch as author)%%%%%%%%%%%%%%%%%%%%%%%%%%%%%%%%%%
\serieslogo{}%please provide filename (without suffix)
\volumeinfo%(easychair interface)
  {Billy Editor and Bill Editors}% editors
  {2}% number of editors: 1, 2, ....
  {Conference title on which this volume is based on}% event
  {1}% volume
  {1}% issue
  {1}% starting page number
\EventShortName{}
\DOI{10.4230/LIPIcs.xxx.yyy.p}% to be completed by the volume editor
% Editor-only macros::end %%%%%%%%%%%%%%%%%%%%%%%%%%%%%%%%%%%%%%%%%%%%%%%

\begin{document}

\maketitle

\begin{abstract}

\end{abstract}

\section{Video Dynamics And Subspace Geometry}

%Explain the advantages over Fourier: automatically finding harmonics, and also able to represent motion without sinc bleed



\subsection{Video Delay Embeddings}
We use a sliding window through time, also known as a ``delay embedding" in the dynamical systems literature \cite{kantz2004nonlinear}, to capture the dynamics of a periodic video.  More precisely, 

\begin{definition}
Given a discrete video $X[n] \in \mathbb{R}^{W \times H}$, where $n \in \mathbb{Z}^+$ is a discrete time index, the video is {\em periodic} if there exists a $T \in \mathbb{Z}^+$ so that $X[n] = X[n + T]$ for all $t$.
\end{definition}

\begin{definition}
Given a video $X[n] \in \mathbb{R}^{W \times H}$ and a window size $M$, {\em delay embedding} $Y[n]$ is formed as 
\begin{equation}
Y[n] = \left[ \begin{array}{c} X[n] \\ X[n + 1] \\ X[n+2] \\ \vdots \\ X[n + (M-1)] \end{array} \right] \in \mathbb{R}^{W \times H \times M}
\end{equation}
\end{definition}

\begin{figure}[]
	\centering
	\includegraphics[width=0.5\columnwidth]{../JumpingJacks/VideoStackTime.pdf}
	\caption{A depiction of a discrete delay embedding of a video of a woman doing jumping jacks with a sliding window embedding of length $M$}
	\label{fig:VideoDiscreteDelayEmbedding}
\end{figure}

As $n$ varies, $Y[n]$ traces out a samples of a 1-manifold in $\mathbb{R}^{W \times H \times M}$, though for a video of $F$ frames, it lies on a $F-1$ dimensional subspace, which we exploit to speed up processing. Figure~\ref{fig:VideoDiscreteDelayEmbedding} shows a pictorial depiction of this scheme.

\subsection{Hypertorus Video Model}
\label{sec:VideoModel}

%\begin{itemize}
%\item Define a fixed period $T \in \mathbb{R^+}$ and a fixed phase shift $\phi \in \mathbb{R}$
%\item Let $I$ be some indexing set into the $N$ pixels of a video frame $[n]$, and let $X_i[n]$, $i \in I$, be a 1-parameter function describing how a pixel at location $i$ evolves in time
%\item Let $g_i(t)$ be an arbitrary 1D function associated with pixel $i$
%\item Let $A$ be a constant associated with amplitude of oscillation.
%\end{itemize}

We now characterize the high dimensional geometry of sliding window embeddings of periodic videos, following a similar analysis to recent work that was done showing that delay embeddings of 1D time series of periodic processes \cite{perea2013sliding}.  We start by assuming very general model for periodic videos.  For a period $T$ and for constants $A$ and $\phi$, and for an arbitrary function $g_i$ at each pixel $X_i$, define the grayscale level at pixel $X_i$ as 

\begin{equation}
X_i[n] = g_i \left( A \cos \left( \frac{2 \pi}{T} n + \phi \right) \right)
\end{equation}

That is, each pixel is an arbitrary function composed with a scale of the same cosine.  Though the function at each pixel may differ, the functions across all pixels are globally {\em in phase}.  This means that the model has mirror symmetry built in.  In particular,

\begin{equation}
X_i[n] = X_i\left[ T - \left(n + \phi \frac{T}{\pi} \right) \right]
\end{equation}

That is, each pixel repeats itself during the second half of its period, but in reverse.  This means that a raw embedding (sliding window size of 1) traces out a curved path between two states $X_a$ and $X_b$.  During the first half of the period the path goes between $X_a$ and $X_b$, and during the second half of the period it follows the {\em exact same} path but in reverse from $X_b$ to $X_a$.  On the other hand, a sliding window size of appropriate length can turn this path into a topological loop by taking a different trajectory from $X_b$ to $X_a$ than was taken from $X_a$ to $X_b$.  A similar observation was made in early work on video textures \cite{schodl2000video}.

To analyze the geometry of the sliding window loops in more detail, express each pixel as a discrete cosine transform with $T$ terms, which is sufficient to summarize it over its period.  Storing all $T$ terms for all $N$ pixels in a period in the $N \times T$ matrix $D$

\begin{equation}
X_i[n] = \sum_{k = 0}^{T-1} D_{ik} \cos \left( \frac{2 \pi}{T} k n  \right)
\end{equation}

With this expansion, all pixels can easily be combined into a vector of the following form:

\begin{equation}
\label{eq:rawexpansion}
X[n] = \sum_{k = 0}^{T-1} \cos \left( \frac{2 \pi}{T} k n \right) D_k
\end{equation}

where $D_k$ is the $k^{\text{th}}$ column of the matrix of DCT coefficients.  A sliding window of length $M$ then takes the following form:

\begin{equation}
Y[n] = \sum_{k = 0}^{T-1} \left[ \begin{array}{c} D_k \cos \left( \frac{2 \pi}{T} k n \right)  \\ D_k \cos \left( \frac{2 \pi}{T} k (n+1) \right) \\ D_k \cos \left( \frac{2 \pi}{T} k (n+2) \right) \\  \vdots \\ D_k \cos \left( \frac{2 \pi}{T} k (n+M-1) \right) \end{array} \right] 
\end{equation}


using the cosine sum identity, this can be rewritten as

\begin{equation}
Y[n] = \sum_{k = 0}^{T-1} \cos \left( \frac{2 \pi}{T} k n \right) \left[ \begin{array}{c} D_k \\ D_k \cos \left( \frac{2 \pi k}{T} \right) \\ D_k \cos \left( 2 \frac{2 \pi k}{T} \right) \\ D_k \cos \left( 3 \frac{2 \pi k}{T} \right) \\ \vdots \\ D_k \cos \left( (M-1) \frac{2 \pi k}{T} \right) \end{array} \right] + \sin \left( \frac{2 \pi}{T} k n \right) \left[ \begin{array}{c} 0_N \\ D_k \sin \left( \frac{2 \pi k}{T} \right) \\ D_k \sin \left( 2 \frac{2 \pi k}{T} \right) \\ D_k \sin \left( 3 \frac{2 \pi k}{T} \right) \\ \vdots \\ D_k \sin \left( (M-1) \frac{2 \pi k}{T} \right) \end{array} \right]
\end{equation}

%Define the vectors $V^s_k$ and $V^c_k$ so that the above equation reads

%\begin{equation}
%Y[n] = \sum_{k = 0}^{T-1} \cos \left( \frac{2 \pi}{T} k n \right) V^c_k + \sin \left( \frac{2 \pi}{T} k n \right) V^s_k
%\end{equation}

Hence, the path that is traced out varying $n$ is the sum of $d \leq T/2$ independent ellipses, which lives on a topological $d$-torus, corresponding to $2d$ nonzero columns in $D$.  In other words, each sinusoidal mode in the video adds an ellipse on a linearly independent plane.  \cite{perea2013sliding} follows a similar line of reasoning to show that delay embeddings of 1D periodic signals live on a hypertorus.


\section{Circular Coordinates}


We use cohomology circular coordinates to parameterize the motion of the video embeddings \cite{de2011persistent}.

To show that there indeed is a periodic process going on, we use phase-based video motion amplification (\cite{wadhwa2013phase}) to amplify all motions within the frequency band consistent with the parameterization found 

\begin{figure}[]
	\centering
	\includegraphics[width=\textwidth]{JumpingJacksPCADGMCC.pdf}
	\caption{Jumping Jacks}
	\label{fig:JumpingJacks}
\end{figure}

\begin{figure}[]
	\centering
	\includegraphics[width=\textwidth]{BeatingHeartSyntheticPCADGMCC.pdf}
	\caption{Beating heart synthetic}
	\label{fig:JumpingJacks}
\end{figure}


\begin{figure}[]
	\centering
	\includegraphics[width=\textwidth]{NeckBeatPCADGMCC.pdf}
	\caption{Neck beating}
	\label{fig:NeckBeating}
\end{figure}

%TODO: Show persistence diagrams, loop versus non-loop, and cohomology circular coordinates


%%
%% Bibliography
%%

%% Either use bibtex (recommended), but commented out in this sample

\bibliography{SlidingWindowVideo}



\end{document}
