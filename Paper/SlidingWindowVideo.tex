\documentclass[a4paper,UKenglish]{lipics}
%This is a template for producing LIPIcs articles. 
%See lipics-manual.pdf for further information.
%for A4 paper format use option "a4paper", for US-letter use option "letterpaper"
%for british hyphenation rules use option "UKenglish", for american hyphenation rules use option "USenglish"
% for section-numbered lemmas etc., use "numberwithinsect"
 
\usepackage{microtype}%if unwanted, comment out or use option "draft"

%\graphicspath{{./graphics/}}%helpful if your graphic files are in another directory

\bibliographystyle{plain}% the recommended bibstyle

% Author macros::begin %%%%%%%%%%%%%%%%%%%%%%%%%%%%%%%%%%%%%%%%%%%%%%%%
\title{High Dimensional Geometry of Sliding Window Embeddings of Periodic Videos}

\author[1]{Christopher J. Tralie}
\affil[1]{Department of Electrical and Computer Engineering, Duke University\\
  Durham, NC USA
  \texttt{chris.tralie@gmail.com}}
\authorrunning{C.\,J Tralie} %mandatory. First: Use abbreviated first/middle names. Second (only in severe cases): Use first author plus 'et. al.'

\Copyright{Christopher J. Tralie}%mandatory, please use full first names. LIPIcs license is "CC-BY";  http://creativecommons.org/licenses/by/3.0/

\subjclass{I.2.10 Video Analysis, I.5.4 Pattern Recognition: Waveform Analysis, I.4.10 Image Representation: Multidimensional}% mandatory: Please choose ACM 1998 classifications from http://www.acm.org/about/class/ccs98-html . E.g., cite as "F.1.1 Models of Computation". 
\keywords{Video Processing, High Dimensional Geometry, Circular Coordinates, Nonlinear Time Series}% mandatory: Please provide 1-5 keywords
% Author macros::end %%%%%%%%%%%%%%%%%%%%%%%%%%%%%%%%%%%%%%%%%%%%%%%%%

%Editor-only macros:: begin (do not touch as author)%%%%%%%%%%%%%%%%%%%%%%%%%%%%%%%%%%
\serieslogo{}%please provide filename (without suffix)
\volumeinfo%(easychair interface)
  {Billy Editor and Bill Editors}% editors
  {2}% number of editors: 1, 2, ....
  {Conference title on which this volume is based on}% event
  {1}% volume
  {1}% issue
  {1}% starting page number
\EventShortName{}
\DOI{10.4230/LIPIcs.xxx.yyy.p}% to be completed by the volume editor
% Editor-only macros::end %%%%%%%%%%%%%%%%%%%%%%%%%%%%%%%%%%%%%%%%%%%%%%%

\begin{document}

\maketitle

\begin{abstract}
We explore the high dimensional geometry of sliding windows of periodic videos. Under a reasonable model for periodic videos, we show that the sliding window is necessary to disambiguate all states within a period, and we show that a video embedding with a sliding window of an appropriate dimension lies on a topological loop along a hypertorus, and that this hypertorus has an independent ellipse for each harmonic of the motion.  Natural motions with sharp transitions from foreground to background have many harmonics and are hence in higher dimensions, so PCA visualizations do not accurately summarize the geometry of these videos.  Noting this, we invoke tools from topological data analysis and cohomology to parameterize motions in high dimensions with circular coordinates after the embeddings.  We show applications to videos in which there is obvious periodic motion, as well as videos in which the motion is hidden.
\end{abstract}

\section{Video Dynamics And Subspace Geometry}

%Explain the advantages over Fourier: automatically finding harmonics, and also able to represent motion without sinc bleed



\subsection{Video Delay Embeddings}
We use a sliding window through time, also known as a ``delay embedding" in the dynamical systems literature \cite{kantz2004nonlinear}, to capture the dynamics of a periodic video.  More precisely, 

\begin{definition}
Given a discrete video $X[n] \in \mathbb{R}^{W \times H}$, where $n \in \mathbb{Z}^+$ is a discrete time index, the video is {\em periodic} if there exists a $T \in \mathbb{Z}^+$ so that $X[n] = X[n + T]$ for all $t$.
\end{definition}

\begin{definition}
Given a video $X[n] \in \mathbb{R}^{W \times H}$ and a window size $M$, {\em delay embedding} $Y[n]$ is formed as 
\begin{equation}
Y[n] = \left[ \begin{array}{c} X[n] \\ X[n + 1] \\ X[n+2] \\ \vdots \\ X[n + (M-1)] \end{array} \right] \in \mathbb{R}^{W \times H \times M}
\end{equation}
\end{definition}

\begin{figure}[]
	\centering
	\includegraphics[width=0.5\columnwidth]{../JumpingJacks/VideoStackTime.pdf}
	\caption{A depiction of a discrete delay embedding of a video of a woman doing jumping jacks with a sliding window embedding of length $M$}
	\label{fig:VideoDiscreteDelayEmbedding}
\end{figure}

As $n$ varies, $Y[n]$ traces out a samples of a 1-manifold in $\mathbb{R}^{W \times H \times M}$, though for a video of $F$ frames, it lies on a $F-1$ dimensional subspace, which we exploit to speed up processing. Figure~\ref{fig:VideoDiscreteDelayEmbedding} shows a pictorial depiction of this scheme.

\subsection{Hypertorus Video Model}
\label{sec:VideoModel}

\begin{figure}[]
	\centering
	\includegraphics[width=0.5\textwidth]{JumpingJacksPCs.png}
	\caption{XT slices of the principal components of a sliding window of length 34 on the jumping jacks videos.  The green line on the left image shows the X slice that is represented in the plots.  Each row corresponds to the two axes of an independent ellipse in the delay embedding}
	\label{fig:JumpingJacksPCs}
\end{figure}

%\begin{itemize}
%\item Define a fixed period $T \in \mathbb{R^+}$ and a fixed phase shift $\phi \in \mathbb{R}$
%\item Let $I$ be some indexing set into the $N$ pixels of a video frame $[n]$, and let $X_i[n]$, $i \in I$, be a 1-parameter function describing how a pixel at location $i$ evolves in time
%\item Let $g_i(t)$ be an arbitrary 1D function associated with pixel $i$
%\item Let $A$ be a constant associated with amplitude of oscillation.
%\end{itemize}

We now characterize the high dimensional geometry of sliding window embeddings of periodic videos, following a similar analysis to recent on delay embeddings of 1D time series \cite{perea2013sliding} (which we also summarize in our video).  We start by assuming very general model for periodic videos.  For a period $T$ and for constants $A$ and $\phi$, and for an arbitrary function $g_i$ at each pixel $X_i$, define the grayscale level at pixel $X_i$ as 

\begin{equation}
X_i[n] = g_i \left( A \cos \left( \frac{2 \pi}{T} n + \phi \right) \right)
\end{equation}

That is, each pixel is an arbitrary function composed with a scale of the same cosine.  Though the function at each pixel may differ, the functions across all pixels are globally {\em in phase}.  This means that the model has mirror symmetry built in.  In particular $X_i[n] = X_i\left[ T - \left(n + \phi \frac{T}{\pi} \right) \right]$.  That is, each pixel repeats itself during the second half of its period, but in reverse, making it is impossible to disambiguate ``going there" from ``coming back."  On the other hand, a sliding window size of appropriate length can turn this path into a topological loop by taking a different trajectory from $X_b$ to $X_a$ than was taken from $X_a$ to $X_b$.  A similar observation was made in early work on video textures \cite{schodl2000video}.  To see this, express each pixel as a discrete cosine transform with $T$ terms, which is sufficient to summarize it over its period.  Storing all $T$ terms for all $N$ pixels in a period in the $N \times T$ matrix $D$,  all pixels can easily be combined into a column vector of the following form:



\begin{equation}
\label{eq:rawexpansion}
X[n] = \sum_{k = 0}^{T-1} \cos \left( \frac{2 \pi}{T} k n \right) D^k
\end{equation}

where $D^k$ is the $k^{\text{th}}$ column of the matrix of DCT coefficients.  A sliding window of length $M$ then takes the following form:

\begin{equation}
Y[n] = \sum_{k = 0}^{T-1} \left[ \begin{array}{c} D^k \cos \left( \frac{2 \pi}{T} k n \right)  \\ D^k \cos \left( \frac{2 \pi}{T} k (n+1) \right) \\ D^k \cos \left( \frac{2 \pi}{T} k (n+2) \right) \\  \vdots \\ D^k \cos \left( \frac{2 \pi}{T} k (n+M-1) \right) \end{array} \right] 
\end{equation}


using the cosine sum identity, this can be rewritten as

\begin{equation}
Y[n] = \sum_{k = 0}^{T-1} \cos \left( \frac{2 \pi}{T} k n \right) \left[ \begin{array}{c} D^k \\ D^k \cos \left( \frac{2 \pi k}{T} \right) \\ D^k \cos \left( 2 \frac{2 \pi k}{T} \right) \\ D^k \cos \left( 3 \frac{2 \pi k}{T} \right) \\ \vdots \\ D^k \cos \left( (M-1) \frac{2 \pi k}{T} \right) \end{array} \right] + \sin \left( \frac{2 \pi}{T} k n \right) \left[ \begin{array}{c} 0^N \\ D^k \sin \left( \frac{2 \pi k}{T} \right) \\ D^k \sin \left( 2 \frac{2 \pi k}{T} \right) \\ D^k \sin \left( 3 \frac{2 \pi k}{T} \right) \\ \vdots \\ D^k \sin \left( (M-1) \frac{2 \pi k}{T} \right) \end{array} \right]
\end{equation}

%Define the vectors $V^s_k$ and $V^c_k$ so that the above equation reads

%\begin{equation}
%Y[n] = \sum_{k = 0}^{T-1} \cos \left( \frac{2 \pi}{T} k n \right) V^c_k + \sin \left( \frac{2 \pi}{T} k n \right) V^s_k
%\end{equation}

\begin{figure}[]
	\centering
	\includegraphics[width=\textwidth]{JumpingJacksPCADGMCC.pdf}
	\caption{Sliding window embedding of a woman doing jumping jacks}
	\label{fig:JumpingJacks}
\end{figure}

Hence, the path that is traced out varying $n$ is the sum of $d \leq T/2$ independent ellipses, each spanned by a plane.  Such paths live on a topological $d$-torus, corresponding to $2d$ nonzero columns in $D$, with 2 dimensions for each independent ellipse.  Also, axes in the embedding space are themselves videos with $T$ frames.  Figure~\ref{fig:JumpingJacksPCs} shows an XT slice of the first 8 principal components of the jumping jacks delay embedding.  Lower frequency ellipse axes correspond to smooth sinusoidal motions, while higher axes correspond to higher harmonics.


\section{Circular Coordinates}



\begin{figure}[]
	\centering
	\includegraphics[width=\textwidth]{BeatingHeartSyntheticPCADGMCC.pdf}
	\caption{Sliding window embedding of a heartbeat animation}
	\label{fig:BeatingHeart}
\end{figure}



\begin{figure}[]
	\centering
	\includegraphics[width=\textwidth]{NeckBeatPCADGMCC.pdf}
	\caption{Sliding window embedding of a video of a person sitting still, which has hidden periodic motion due to the person's heartbeat}
	\label{fig:NeckBeating}
\end{figure}



%.  From right to left: An example point in the sliding window embedding made up of a stack of frames, 3D PCA of all sliding windows, the 1D persistence diagram of the Rips cohomology filtration, and the circular coordinates corresponding to the strongest 1D homology class

Now that we know that the sliding window embeddings of videos lie on a highly curved topological loop on a hypertorus, we can use high dimensional data analysis tools to extract information about these loops.  We turn to the theory of 1D persistent homology and Rips Complexes to measure the geometric prominence of the loops in high dimensions.  In particular, we use cohomology circular coordinates to parameterize the motion of the video embeddings \cite{de2011persistent}.  Figure~\ref{fig:JumpingJacks} shows an example of applying 1D Rips and persistent cohomology to extract circular coordinates for the jumping jack example.  The loop is visible with 3D PCA, but very little of the variance is explained by the first 3 components since this video has sharp transitions from foreground to background, which need to be represented with many harmonics, as shown in our video.  But the circular coordinates are able to parameterize the motion in high dimensions.  A similar pattern is visible for a synthetic beating heart video in Figure~\ref{fig:BeatingHeart}.  We can also apply our techniques to videos with very subtle motions.  Figure~\ref{fig:NeckBeating} shows such an example with a person sitting still in front of a camera.  Hardly any motion is visible, but the persistence diagram and circular coordinates indicate the presence of a cycle.  In fact, this cycle corresponds to twice the person's heartbeat, which exists as a low magnitude vibration in the video.  To show that there indeed is a periodic process going on, we use phase-based video motion amplification (\cite{wadhwa2013phase}) to amplify all motions within the frequency band consistent with the parameterization found.  As a next step, we are currently working on other examples in which there are two possibly non-commensurate periodic processes, such as heartbeat and breathing, with the goal of separating out these motions using cohomology.

%\begin{figure}[]
%	\centering
%	\includegraphics[width=0.7\textwidth]{JumpingJacksPCAPDRaw.pdf}
%	\caption{Jumping Jacks raw}
%	\label{fig:JumpingJacksRaw}
%\end{figure}




%TODO: Show persistence diagrams, loop versus non-loop, and cohomology circular coordinates


%%
%% Bibliography
%%

%% Either use bibtex (recommended), but commented out in this sample

\bibliography{SlidingWindowVideo}



\end{document}
